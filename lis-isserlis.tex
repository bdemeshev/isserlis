% arara: xelatex
\documentclass[12pt]{article}

% \usepackage{physics}

\usepackage{hyperref}
\hypersetup{
    colorlinks=true,
    linkcolor=blue,
    filecolor=magenta,      
    urlcolor=cyan,
    pdftitle={Overleaf Example},
    pdfpagemode=FullScreen,
    }

\usepackage{tikzducks}

\usepackage{tikz} % картинки в tikz
\usetikzlibrary{shapes, arrows, positioning}
\usepackage{microtype} % свешивание пунктуации

\usepackage{array} % для столбцов фиксированной ширины

\usepackage{indentfirst} % отступ в первом параграфе

\usepackage{sectsty} % для центрирования названий частей
\allsectionsfont{\centering}

\usepackage{amsmath, amsfonts, amssymb} % куча стандартных математических плюшек

\usepackage{comment}

\usepackage[top=2cm, left=1.2cm, right=1.2cm, bottom=2cm]{geometry} % размер текста на странице

\usepackage{lastpage} % чтобы узнать номер последней страницы

\usepackage{enumitem} % дополнительные плюшки для списков
%  например \begin{enumerate}[resume] позволяет продолжить нумерацию в новом списке
\usepackage{caption}

\usepackage{url} % to use \url{link to web}


\newcommand{\smallduck}{\begin{tikzpicture}[scale=0.3]
    \duck[
        cape=black,
        hat=black,
        mask=black
    ]
    \end{tikzpicture}}

\usepackage{fancyhdr} % весёлые колонтитулы
\pagestyle{fancy}
\lhead{}
\chead{Хитрый Лис Иссерлис}
\rhead{}
\lfoot{}
\cfoot{}
\rfoot{}

\renewcommand{\headrulewidth}{0.4pt}
\renewcommand{\footrulewidth}{0.4pt}

\usepackage{tcolorbox} % рамочки!

\usepackage{todonotes} % для вставки в документ заметок о том, что осталось сделать
% \todo{Здесь надо коэффициенты исправить}
% \missingfigure{Здесь будет Последний день Помпеи}
% \listoftodos - печатает все поставленные \todo'шки


% более красивые таблицы
\usepackage{booktabs}
% заповеди из докупентации:
% 1. Не используйте вертикальные линни
% 2. Не используйте двойные линии
% 3. Единицы измерения - в шапку таблицы
% 4. Не сокращайте .1 вместо 0.1
% 5. Повторяющееся значение повторяйте, а не говорите "то же"


\setcounter{MaxMatrixCols}{20}
% by crazy default pmatrix supports only 10 cols :)


\usepackage{fontspec}
\usepackage{libertine}
\usepackage{polyglossia}

\setmainlanguage{russian}
\setotherlanguages{english}

% download "Linux Libertine" fonts:
% http://www.linuxlibertine.org/index.php?id=91&L=1
% \setmainfont{Linux Libertine O} % or Helvetica, Arial, Cambria
% why do we need \newfontfamily:
% http://tex.stackexchange.com/questions/91507/
% \newfontfamily{\cyrillicfonttt}{Linux Libertine O}

\AddEnumerateCounter{\asbuk}{\russian@alph}{щ} % для списков с русскими буквами
% \setlist[enumerate, 2]{label=\asbuk*),ref=\asbuk*}

%% эконометрические сокращения
\DeclareMathOperator{\Cov}{\mathbb{C}ov}
\DeclareMathOperator{\Corr}{\mathbb{C}orr}
\DeclareMathOperator{\Var}{\mathbb{V}ar}
\DeclareMathOperator{\col}{col}
\DeclareMathOperator{\row}{row}

\let\P\relax
\DeclareMathOperator{\P}{\mathbb{P}}

\DeclareMathOperator{\E}{\mathbb{E}}
% \DeclareMathOperator{\tr}{trace}
\DeclareMathOperator{\card}{card}
\DeclareMathOperator{\mgf}{mgf}

\DeclareMathOperator{\Convex}{Convex}
\DeclareMathOperator{\plim}{plim}

\usepackage{mathtools}
\DeclarePairedDelimiter{\norm}{\lVert}{\rVert}
\DeclarePairedDelimiter{\abs}{\lvert}{\rvert}
\DeclarePairedDelimiter{\scalp}{\langle}{\rangle}
\DeclarePairedDelimiter{\ceil}{\lceil}{\rceil}

\newcommand{\cN}{\mathcal{N}}
\newcommand{\cF}{\mathcal{F}}

\newcommand{\RR}{\mathbb{R}}
\newcommand{\NN}{\mathbb{N}}
\newcommand{\hb}{\hat{\beta}}
\newcommand{\dPois}{\mathrm{Pois}}





\begin{document}
Цель этой заметки — доказать теорему Иссерлиса о подсчёте ожиданий для многомерного нормального распределения.
По дороге выведем функцию производящую моменты для одномерного нормального $\cN(\mu, \sigma^2)$ и многомерного нормального $\cN(\mu, C)$.

\section*{Ожидание экспоненты}
Первая задача. 
Найдите $\E(\exp(Y))$ для нормальной $Y \sim \cN(\mu; \sigma^2)$.

Сначала стандартизируем $Y$, $Y = \mu + \sigma X$, где $X \sim \cN(0; 1)$:
\[
\E(\exp(Y)) = \E(\exp(\mu)\exp(\sigma X)) = \exp(\mu) \E(\exp(\sigma X)).
\]
Перейдём к интегралам!
\[
\E(\exp(\sigma X)) = \int_{\RR} \exp(\sigma x) f(x) dx = \int_{\RR} \exp(\sigma x) \frac{1}{\sqrt{2\pi}} \exp(-x^2/2) dx.
\]
Для взятия интеграла выделим полный квадрат внутри экспоненты:
\[
\sigma x - x^2/2 = -\frac{1}{2} (x^2 - 2\sigma x  + \sigma^2 - \sigma^2) = -\frac{1}{2} (x - \sigma)^2 + \frac{1}{2} \sigma^2.
\]
Возвращаемся к интегралу:
\[
\E(\exp(\sigma X)) = \dots = \int_{\RR} \exp(\sigma^2/2) \frac{1}{\sqrt{2\pi}} \exp(-(x- \sigma)^2/2) dx = \exp(\sigma^2/2) \int_{\RR} \frac{1}{\sqrt{2\pi}} \exp(-(x- \sigma)^2/2) dx.
\]
Замечаем, что последний интеграл — это площадь под нормальной функцией плотности, смещённой на $\sigma$ вправо. 
И эта площадь равна единице. 

Задача раз решена,
\[
    \E(\exp(Y)) = \exp(\mu) \exp(\sigma^2/2) = \exp(\mu + \sigma^2/2).
\]

\section*{Одномерная функция производящая моменты}
Вторая задача. 
Найдите функцию производящую моменты для $Y \sim \cN(\mu, \sigma^2)$.

По-определению, $\mgf_Y(u) = \E(\exp(uY))$.

Заметим, что $uY \sim \cN(\mu \cdot u, \sigma^2 \cdot u^2)$.

Поэтому сразу получаем, что $\mgf_Y(u) = \exp(\mu u + \sigma^2 u^2 /2)$.

\section*{Многомерная функция производящая моменты}
Третья задача. 
Найдите функцию производящую моменты для случайного вектора $Y \sim \cN(\mu, C)$.

По-определению, $\mgf_Y(u) = \E(\exp(u^T Y))$.

Заметим, что $u^T Y$ это скалярная случайная величина с нормальным распределением $\cN(u^T \mu, u^T C u)$.

Снова быстро получаем функцию производящую моменты, 
\[
\mgf_Y(u) = \exp(u^T\mu + u^T C u/2).
\]
Производящая функция выглядит как экспоненты от квадратичной функции,
\[
\mgf_Y(u) = \exp(q(u)), \quad q(u) = u^T\mu + u^T C u/2.
\]

\section*{Теорема Иссерлиса}
Четвёртая задача. 
Для случайного вектора $Y \sim \cN(\mu, C)$ последовательно найдите $\E(Y_1)$, $\E(Y_1 Y_2)$, $\E(Y_1 Y_2 Y_3)$ и $\E(Y_1 Y_2 Y_3 Y_4)$.

Вспомним, что $\E(Y_1) = \mgf'_1(0)$, $\E(Y_1 Y_2) = \mgf''_{12}(0)$ и так далее. 

Немного заранее подготовимся! Во-первых, в нуле $q(0) = 0$, $\exp(q(0)) = 1$.

Найдём первую производную $q'_1(u) = \mu_1 + c_1^T u$, где $c_1$ — первый столбец матрицы $C$.
Для наглядности перепишем её в скалярном виде
\[
    q'_1(u_1, u_2, \dots, u_n) = \mu_1 + c_{11}u_1 + c_{12}u_2 + \dots + c_{1n}u_n.
\]
В нуле первая производная равна $q'_1(0) = \mu_1$ соответствующему ожиданию. 

Вторая производная $q''_{12}(u) = c_{12}$ тождественно равна соответствующей ковариации. 

Третья производная $q'''_{123} = 0$  тождественно равна нулю, ведь $q(u)$ — квадратичная функция.

А теперь считаем ожидания по очереди,
\[
    \E(Y_1) = \mgf'_1 = \exp(q) q'_1 = 1 \cdot \mu_1 = \mu_1.
\]
Пока что ничего неожиданного, мы же сами обозначили $\E(Y_1)$ как $\mu_1$.

Пойдём дальше!
\[
    \E(Y_1Y_2) = \mgf''_{12} = \exp(q) (q'_1q'_2 + q''_{12}) = \mu_1 \mu_2 + c_{12}.
\]
Это тождество, верное для любый случайных величин, не только для нормальных, $\E(Y_1 Y_2) = \E(Y_1) \E(Y_2) + \Cov(Y_1, Y_2)$.

Продолжаем,
\[
    \E(Y_1 Y_2 Y_3) = \mgf'''_{123} = \exp(q) (q'_1 q'_2 q'_3 + q'_3 q''_{12} + q'_1 q''_{23} + q'_2 q''_{13}) = \mu_1 \mu_2\mu_3 + \mu_1 c_{23} + \mu_2 c_{13} + \mu_3 c_{12}.
\]
\begin{tcolorbox}[colback=yellow!50!red!25!white]
Каждое слагаемое идёт с единичным весом.
Каждое слагаемое содержит все индексы от единицы до максимального ровно по одному разу. 
Все варианты перемножения $\mu_i$ и $c_{jk}$ присутствуют. 
Множитель $\mu_i$ «съедает» один индекс, а $c_{jk}$ «съедает» два индекса. 
\end{tcolorbox}

Если какой-то индекс повторяется, то его надо повторить :)
\[
    \E(Y_1 Y_2 Y_2) = \mgf'''_{122} = \exp(q) (q'_1 q'_2 q'_2 + q'_2 q''_{12} + q'_1 q''_{22} + q'_2 q''_{12}) = \mu_1 \mu_2\mu_2 + \mu_1 c_{22} + \mu_2 c_{12} + \mu_2 c_{12}.
\]

Фанаты могут убедиться, что 
\[
    \E(Y_1 Y_2 Y_3 Y_4) = \mu_1 \mu_2\mu_3 \mu_4 + \mu_1 \mu_4 c_{23} + \mu_2 \mu_4 c_{13} + \mu_3 \mu_4 c_{12} + \mu_1 \mu_3 c_{24} + \mu_1 \mu_2 c_{34} + \mu_2 \mu_3 c_{14} +  c_{12}c_{34} + c_{13}c_{24} + c_{14}c_{23}.
\]


Пример.

Вектор $Y$ имеет совместное нормальное распределение,
\[
\begin{pmatrix}
    Y_1 \\
    Y_2 \\
    Y_3 \\
\end{pmatrix} \sim \cN\left(
\begin{pmatrix}
    1 \\
    2 \\
    3 \\
\end{pmatrix},
\begin{pmatrix}
    5 & -1 & -2 \\
    -1 & 6 & -3 \\
    -2 & -3 & 7 \\
\end{pmatrix}
\right).
\]

Найдите $\E(Y_1 Y_2 Y_3)$.


Решение:
\[
    \E(Y_1 Y_2 Y_3)  = \mu_1 \mu_2\mu_3 + \mu_1 c_{23} + \mu_2 c_{13} + \mu_3 c_{12} = 1\cdot 2 \cdot 3 + 1 \cdot (-3) + 2 \cdot (-2) + 3\cdot (-1) = -2.
\]


\end{document}

